\documentclass[dvipdfmx,cjk]{beamer} 
\AtBeginDvi{\special{pdf:tounicode 90ms-RKSJ-UCS2}} 
\usetheme{CambridgeUS} 
\usecolortheme{seahorse}
\setbeamertemplate{theorems}[numbered]
\newtheorem{thm}{Theorem}[section]
\newtheorem{proposition}[thm]{Proposition}
\theoremstyle{example}
\newtheorem{exam}[thm]{Example}
\newtheorem{remark}[thm]{Remark}
\newtheorem{question}[thm]{Question}
\newtheorem{prob}[thm]{Problem}
\begin{document}
\title[Chap8]{Chapter.8 決定論的微分方程式の離散時間近似} 
\author[Katsuya ITO]{伊藤克哉} 
\institute[UT]{東京大学}
\date{2016/09/26}
\begin{frame}                  %% \begin{frame}..\end{frame} で 1 枚のスライド
\titlepage                     %% タイトルページ
\end{frame}

\begin{frame}                  %% 目次 (必要なければ省略)
\tableofcontents
\end{frame}

\section{箇条書き}             %% セクション名
\begin{frame}
\frametitle{松本}              %% フレームタイトル

\begin{itemize}
\item 豊科\pause               %% \pause でとまる
\item 穂高\pause
\item 明科
\end{itemize}
\end{frame}

\section{定理型環境}           %% 定理型環境が使える
\begin{frame}                  %% \newtheorem で新しい環境も作れる
\begin{thm}
定理型環境が使える。
使い方は普通の \LaTeX と同じ
\end{thm}
\pause

\begin{proof}
証明も書ける。
\end{proof}
\pause

\begin{exam}                   %% 色が違う
example
\end{exam}
\end{frame}

\section{文字の色}             %% 文字の色を変える
\begin{frame}
\frametitle{文字の色を変えてみよう}
{\color{red}赤}\pause
{\color{blue}青}\pause
{\color{green}緑}
\end{frame}

\end{document}